\documentclass[fontsize=10pt, twoside=no]{scrartcl} % KOMA class

% other packages %
\usepackage{graphicx}
\usepackage{titlesec}
%\usepackage{url}
\usepackage{fancybox}

% lang : french %
\usepackage[utf8]{inputenc}
\usepackage{xspace}
\usepackage[T1]{fontenc}
\usepackage[english,frenchb]{babel}

% mise en page %
\KOMAoptions{parskip=half+}
\KOMAoptions{paper=a4,DIV=22}

\addto\captionsfrench{\def\partname{}}
\renewcommand{\thepart}{}
%%%%%%%%%%%%%%

\begin{document}

\begin{titlepage}
\pagestyle{headings}

%\begin{minipage}[c]{\textwidth}
%  \begin{center}
%    \includegraphics [height=60mm]{logo.png} \\[0.5cm]
%  \end{center}
%\end{minipage}

\begin{center}

\thispagestyle{empty}

\vspace*{5\baselineskip}

\textsc{\Large Synthèse d'images}\\[0.2cm]
janvier 2014\\[0.5cm]

\vspace*{2\baselineskip}

\begin{minipage}[t]{.8\textwidth}
  \begin{flushleft} \large
    %\emph{Auteurs :}
    Victor \textsc{Degliame} -
    Florian \textsc{Thomassin}
  \end{flushleft}
\end{minipage}

\vspace*{5\baselineskip}

\begin{minipage}[t]{0.8\textwidth}
  \noindent
  \emph{Sujet :}\\
  Création d'un Ray Tracer 100\% à la main. Seuls les dépendances pour les I/O sont autorisées. Doit obligatoirement
  gérer les lumières ambiantes, un nombre arbitraire de lumières omnidirectionnelles non ponctuelles, gérer le diffus et
  le spéculaire, des textures monochromes pour chaque objet et des surfaces définies par des triangles avec ou sans
  lissage.
\end{minipage}


\end{center}

\end{titlepage}


\title{Ray Tracer}
\author{Victor \textsc{Degliame} - Florian \textsc{Thomassin}}
%\maketitle

\part{Les lumières}

    \section{Lumière ambiante}

C'est une lumière qui est reçue de façon uniforme par tous les points de la scène. Elle ne prend pas en compte la
position des autres sources lumineuses et des réflections sur les objets de la scène. Elle permet d'éviter que certaines
zones soient complètement noires et donne un effet plus réaliste à la scène car il est rare qu'un objet ne recoivent pas
du tout de lumière étant donné que tous les objets réfléchissent un peu de lumière.

    \section{Lumières omnidirectionnelles non ponctuelles}

C'est une source de lumière placée dans un point de l'espace et qui forme une "boule" qui éclaire dans toutes les
directions de façon uniforme. Si un rayon intersecte cette boule alors on considère que le point d'où provient le rayon
est éclairé par cette source de lumière. Une telle source a une influence diffuse et spéculaire sur les objets qu'elle
éclaire.

    \section{Le diffus}

    \section{Le Spéculaire}

\part{Les Textures}

    \section{monochrome}

\part{Représentation des objets}

    \section{Surface définie par triangles}
    
\end{document}
