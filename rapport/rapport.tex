\documentclass[fontsize=10pt, twoside=no]{scrartcl} % KOMA class

% other packages %
\usepackage{graphicx}
\usepackage{titlesec}
%\usepackage{url}
\usepackage{fancybox}

% lang : french %
\usepackage[utf8]{inputenc}
\usepackage{xspace}
\usepackage[T1]{fontenc}
\usepackage[english,frenchb]{babel}

% mise en page %
\KOMAoptions{parskip=half+}
\KOMAoptions{paper=a4,DIV=22}

\addto\captionsfrench{\def\partname{}}
\renewcommand{\thepart}{}
%%%%%%%%%%%%%%

\begin{document}

\begin{titlepage}
\pagestyle{headings}

%\begin{minipage}[c]{\textwidth}
%  \begin{center}
%    \includegraphics [height=60mm]{logo.png} \\[0.5cm]
%  \end{center}
%\end{minipage}

\begin{center}

\thispagestyle{empty}

\vspace*{5\baselineskip}

\textsc{\Large Synthèse d'images}\\[0.2cm]
janvier 2014\\[0.5cm]

\vspace*{2\baselineskip}

\begin{minipage}[t]{.8\textwidth}
  \begin{flushleft} \large
    %\emph{Auteurs :}
    Victor \textsc{Degliame} -
    Florian \textsc{Thomassin}
  \end{flushleft}
\end{minipage}

\vspace*{5\baselineskip}

\begin{minipage}[t]{0.8\textwidth}
  \noindent
  \emph{Sujet :}\\
  Création d'un Ray Tracer 100\% à la main. Seuls les dépendances pour les I/O sont autorisées. Doit obligatoirement
  gérer les lumières ambiantes, un nombre arbitraire de lumières omnidirectionnelles non ponctuelles, gérer le diffus et
  le spéculaire, des textures monochromes pour chaque objet et des surfaces définies par des triangles avec ou sans
  lissage.
\end{minipage}


\end{center}

\end{titlepage}


\title{Ray Tracer}
\author{Victor \textsc{Degliame} - Florian \textsc{Thomassin}}
%\maketitle

\part{Les lumières}

    \section{Lumière ambiante}
    
    \section{Lumières omnidirectionnelles non ponctuelles}

    \section{Le diffus}

    \section{Le Spéculaire}

\part{Les Textures}

    \section{monochrome}

\part{Représentation des objets}

    \section{Surface définie par triangles}
    
\end{document}
