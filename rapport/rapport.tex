\documentclass[fontsize=10pt, twoside=no]{scrartcl} % KOMA class

% other packages %
\usepackage{graphicx}
\usepackage{titlesec}
%\usepackage{url}
\usepackage{fancybox}

% lang : french %
\usepackage[utf8]{inputenc}
\usepackage{xspace}
\usepackage[T1]{fontenc}
\usepackage[english,frenchb]{babel}

% mise en page %
\KOMAoptions{parskip=half+}
\KOMAoptions{paper=a4,DIV=22}

\addto\captionsfrench{\def\partname{}}
\renewcommand{\thepart}{}
%%%%%%%%%%%%%%

\begin{document}

\begin{titlepage}
\pagestyle{headings}

%\begin{minipage}[c]{\textwidth}
%  \begin{center}
%    \includegraphics [height=60mm]{logo.png} \\[0.5cm]
%  \end{center}
%\end{minipage}

\begin{center}

\thispagestyle{empty}

\vspace*{5\baselineskip}

\textsc{\Large Synthèse d'images}\\[0.2cm]
janvier 2014\\[0.5cm]

\vspace*{2\baselineskip}

\begin{minipage}[t]{.8\textwidth}
  \begin{flushleft} \large
    %\emph{Auteurs :}
    Victor \textsc{Degliame} -
    Florian \textsc{Thomassin}
  \end{flushleft}
\end{minipage}

\vspace*{5\baselineskip}

\begin{minipage}[t]{0.8\textwidth}
  \noindent
  \emph{Sujet :}\\
  Création d'un Ray Tracer 100\% à la main. Seuls les dépendances pour les I/O sont autorisées. Doit obligatoirement
  gérer les lumières ambiantes, un nombre arbitraire de lumières omnidirectionnelles non ponctuelles, gérer le diffus et
  le spéculaire, des textures monochromes pour chaque objet et des surfaces définies par des triangles avec ou sans
  lissage.
\end{minipage}


\end{center}

\end{titlepage}


\title{Ray Tracer}
\author{Victor \textsc{Degliame} - Florian \textsc{Thomassin}}
%\maketitle

\part{Les lumières}

    \section{Lumière ambiante}

C'est une lumière qui est reçue de façon uniforme par tous les points de la scène. Elle ne prend pas en compte la
position des autres sources lumineuses et des réflections sur les objets de la scène. Elle permet d'éviter que certaines
zones soient complètement noires et donne un effet plus réaliste à la scène car il est rare qu'un objet ne recoivent pas
du tout de lumière étant donné que tous les objets réfléchissent un peu de lumière.

    \section{Lumières omnidirectionnelles non ponctuelles}

C'est une source de lumière placée dans un point de l'espace et qui forme une "boule" qui éclaire dans toutes les
directions de façon uniforme. Si un rayon intersecte cette boule alors on considère que le point d'où provient le rayon
est éclairé par cette source de lumière. Une telle source a une influence diffuse et spéculaire sur les objets qu'elle
éclaire.

    \section{Le diffus}

Cela représente la lumière qui s'étale sur la surface d'un objet. Plus le rayon lumineux est perpendiculaire à la
tangente à l'objet au point d'impact, plus l'énergie du faisceau lumineux réstituée est importante. Ainsi si un rayon
lumineux arrive sur un objet de pleine face, l'énergie restituée sera grande tandis que si le rayon arrive de façon
tangentielle à l'objet, l'energie restituée sera très faible. La formule permettant de calculer la couleur du rayon
diffusé est la suivante :

$$C_{r} \mathrel{+}= \rho _{d} \times C_{l} \times \vec{L} \cdot \vec{N}$$

Avec $C_{r}$ la couleur actuelle du rayon de la caméra. $\rho _{d}$ le taux de diffusion spécifique à chaque objet qui détermine à quel
point la capacité d'un objet à diffuser la lumière est importante. $C_{l}$ la couleur de la lumière. $\vec{L}$ le rayon
d'incidence de la lumière. $\vec{N}$ la normale à l'objet au point d'impact.
 

    \section{Le Spéculaire}

La lumière spéculaire quand à elle correspond à la réflexion des rayons lumineux sur un objet réfléchissant. Plus l'oeil
sera en face d'un rayon réfléchi sur un objet, plus l'énergie reçue sera importante. A condition que le taux de
réfléxivité $\rho_{r}$ de l'objet soit assez élevé, bien entendu. La formule de calcul de la couleur du rayon de la caméra
pour le spéculaire est donc :

$$C_{r} \mathrel{+}= \rho _{r} \times C_{l} \times \vec{LRefl} \cdot \vec{R}$$

Avec $C_{r}$ la couleur actuelle du rayon de la caméra. $\rho_{r}$ le taux de réfléxivité de l'objet. $C_{l}$ la couleur
de la lumière. $\vec{LRefl}$ le rayon réfléchi de la lumière sur l'objet. $\vec{R}$ le rayon de la caméra.

\part{Les Textures}

    \section{monochrome}

La texture de tous les triangles est monochrome (en RGB). Ainsi un rayon de la caméra touchant un objet se vera
attribuer la couleur de l'objet. A cela viendra s'ajouter les couleurs provenant des différentes lumières.

\part{Représentation des objets}

    \section{Surface définie par triangles}

Toutes les surfaces sont définies par des triangles car ils sont très facile à manipuler. Ces triangles sont définis par
les coordonnées de leurs trois sommets ainsi que de leur normale qui permet de connaitre leur orientation. Cette normale
va notamment être utilisée pour les calculs de diffusion et de réflexion de la lumière ainsi que pour les effets de
lissage telles Phong ou Gouraud.

Le principe de ces techniques de lissage est assez simple, lorsque dans une de nos formules on utilise la normale à
l'objet impacté par le rayon, à la place, on va prendre une moyenne pondérée de la normale de cet objet et de celles de
ses voisins. Plus le point d'impact est proche d'un triangle, plus sa normale aura d'influence. Ainsi on obtient un
effet lissé beaucoup plus réaliste et il faut moins de polygones pour avoir un rendu correct.
    
\part{Bonus}
\end{document}
